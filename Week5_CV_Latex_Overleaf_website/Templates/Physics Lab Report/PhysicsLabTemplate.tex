\documentclass[11pt, oneside]{article}   	% <-- Could change the font size up here
\usepackage{geometry}                			% Packages and things to be imported that may be used
\geometry{letterpaper}                   			% Later on in tables, figures, etc.
\usepackage{graphicx}				

\usepackage{amssymb}
\usepackage{float}
\usepackage{array}



\title{Physics Lab Template}
\author{\underline{Your Name}, Partner Name}
\date{Tuesday, February 9th, 2015}					

\begin{document} % Starts the Document 
\maketitle % Makes the Title


\abstract % Latex has its own abstract heading!
Abstract section where you discuss what you found, if you wish to type a math equation inside text, use dollar signs on either side of your equation: $g = 9.81m/s^2$




\section{Theory}
Theory section

\subsection{Part 1}
Some specific part of the theory behind the experiment
\newline % Go to the next line
% The next commented section gives the layout to make a figure at the location in the document where it is typed

% \begin{figure}[H]
%    \centering
%    \includegraphics[scale=0.3]{Folder/diagramname.jpg}
%    \caption{This is what figure 1 is}
%    \label{Figure 1}
% \end{figure}
\par Might use equations to to describe something, like % \par does paragraph indentation
\begin{equation} \vec{L} = I \vec{\omega} \end{equation}


\section{Experiment}
Experiment section



\section{Data Analysis}
Might want a table to present data:

\begin{table}[h]
\centering
\begin{tabular}{|c|c|c|}
	\hline
	Trial & 1 & 2  \\ \hline
	$\theta_1$ & 1 & 2 \\ \hline
	$\theta_2$ & 1 & 2 \\ \hline
	\end{tabular}
	\caption{caption about table 1}
	\label{Table 1}
\end{table}

Of course, error propagation will be in here:
\begin{equation}  \sigma = \sqrt{\left(\frac{\partial}{\partial x}\right)^2\sigma _x^2+\left(\frac{\partial}{\partial y}\right)^2\sigma _y^2 + \left(\frac{\partial}{\partial z}\right)^2\sigma _z^2...} \end{equation}

Maybe some weighted mean:
\begin{equation} w_i = \frac{1}{\sigma_i^2}  \end{equation}
\begin{equation} x_{mean} = \frac{\sum\limits_{i=1}^N{w_ix_i}}{\sum\limits_{i=1}^N{w_i}} \end{equation}
\begin{equation} \sigma_{mean} = \frac{1}{\sqrt{\sum\limits_{i=1}^N{w_i}}} \end{equation}


\section{Conclusion}
Summary



\section{Remarks}
remarks




\end{document}  % Ends the masterpiece